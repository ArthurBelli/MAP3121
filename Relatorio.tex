\documentclass{article}

% Language setting
% Replace `english' with e.g. `spanish' to change the document language
\usepackage[portuguese]{babel}

% Set page size and margins
% Replace `letterpaper' with `a4paper' for UK/EU standard size
\usepackage[letterpaper,top=2cm,bottom=2cm,left=3cm,right=3cm,marginparwidth=1.75cm]{geometry}

% Useful packages
\usepackage{amsmath}
\usepackage{graphicx}
\usepackage[colorlinks=true, allcolors=blue]{hyperref}

\title{\textbf{EP1}- Decomposição LU para matrizes tridiagonais \newline \\

 \large MAP3121- Métodos Numéricos e Aplicações \\
}

\author{Arthur Belli - NUSP \newline \\  Letícia Furusawa - 11965585}

\begin{document}
\maketitle


\section{Introdução}

Este relatório descreve os resultados obtidos no exercício programa, no qual estudamos decomposição LU de matrizes e resolução de sistemas lineares para matrizes tridiagonais e cíclicas. 

\section {Funções utilizadas}

\subsection{LU _ to _ A}

Realiza a operação de multiplicação das matrizes L e U e devolve a matriz A; resultado da multiplicação. 
Ela recebe como parâmetros as matrizes L, U e a dimensão dessas matrizes. 



\section{Resultado}

\subsection{Exercício 1}

O objetivo do exercício 1 era provar a seguinte igualdade; resultado da decomposição LU: \newline \\

\begin{eq1}
\centering A_{i,j} =  \sum_{k=1}^{min[i,j]} L_{ik}U_{kj} \ (1)  \newline \\
\end{eq1}

Utilizando as seguintes matrizes de entrada \newline \\

L =  \begin{bmatrix}a & b \\c & d \end{bmatrix} e U = \begin{bmatrix}a & b \\c & d \end{bmatrix} \newline \\

na equação (1), obtemos o seguinte resultado :

\subsection{Exercício 2}

Neste exercício, devemos calcular mas matrizes L e U a partir da matriz A e, usando o resultado da decomposição LU, resolver um sistema linear.  


\subsection{Exercício 3} 

O objetivo desse exercício é solucionar sistemas lineares provenientes de matrizes tridiagonais, utilizando o algoritmo de decomposição LU implementado anteriormente. 

\subsection{Resolução do exemplo fornecido}

\section{Conclusão}


\bibliographystyle{alpha}

\bibliography{sample}

\end{document}
